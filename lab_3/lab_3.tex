\documentclass[12pt]{article}
\usepackage[left=2cm,right=2cm,top=2cm,bottom=2cm,bindingoffset=0cm]{geometry}

\usepackage[utf8]{inputenc}
\usepackage[russian]{babel}

\usepackage{listings}

\begin{document}
\section{Код}
\begin{lstlisting}[language=C,frame=single,numbers=left]
#include <stdio.h>
#include <unistd.h>
#include <sys/wait.h>
#include <time.h>

#define N 1000
#define M 100000000


int ar[M]; # or not

int main() {
    for (int i = 0; i < M; ++i) { # or not
        ar[i] = i;                # or not
    }                             # or not

    clock_t start_time = clock();
    for (int i = 0; i < N; ++i) {
        pid_t pid = fork(); # or vfork
        if (pid) {
            int status;
            waitpid(pid, &status, 0);
        } else {
            execlp("/bin/true", "/bin/true", NULL, NULL);
        }
    }
    clock_t end_time = clock();
    printf("%f\n", (float)(end_time - start_time) / CLOCKS_PER_SEC);
    return 0;
}
\end{lstlisting}
\begin{lstlisting}[language=bash]
gcc main.c
\end{lstlisting}
\section{Выводы}
\begin{lstlisting}[language=bash]
./a.out
\end{lstlisting}
\begin{center}
  \begin{tabular}{ | l | c | r | }
    \hline
     & fork & vfork \\ \hline
    без массива & 0.065288 & 0.032956 \\ \hline
    с массивом & 6.286745 & 0.033586 \\
    \hline
  \end{tabular}
\end{center}
Время указано в секундах на 1000 запросов. Без использования большого количества памяти, использование vfork уменьшает время работы примерно в два раза. В случае "тяжёлых" процессов время работы с использованием vfork не изменяется, а с использованием fork - ухудшается примерно в 100 раз.
\end{document}