\documentclass[12pt]{article}
\usepackage[left=2cm,right=2cm,top=2cm,bottom=2cm,bindingoffset=0cm]{geometry}
\usepackage[utf8]{inputenc}
\usepackage[russian]{babel}

\usepackage{listings}

\begin{document}
\section{Код}
\begin{lstlisting}[language=C,numbers=left,frame=single]
#include <stdio.h>
#include <stdbool.h>
#include <unistd.h>
#include <pthread.h>


void *my_thread_function(void *args) {
    const char *c=(const char *)args;
    while (true) {
        write(STDOUT_FILENO, c, 1); # or printf("%s", c);
    }
    return NULL;
}


int main() {
    pthread_t tid_1, tid_2;

    pthread_create(&tid_1, NULL, my_thread_function, (void *)"a");
    pthread_create(&tid_2, NULL, my_thread_function, (void *)"b");
    sleep(1);

    return 0;
}
\end{lstlisting}
\begin{lstlisting}[language=bash]
gcc main.c -pthread
\end{lstlisting}
\section{Вывод}
\textbf{write} - функция предоставляемя ядром ОС, позволяет записать массив байт в какой-нибудь файл по файловому дискриптеру. STDOUT\_FILENO=1 - дескриптор стандартного потока вывода.
\newline
Вывод: \dots ababab\dots
\newline
\newline
\textbf{printf} - "print formated". Использует буфер. Когда буфер переполняется, имхо, вызывается write. Более быстрая, чем write, ибо не приходится каждый раз делать syscall.
\newline
Вывод: \dots aaa\dots aaa\dots abbb\dots bbb\dots aaa\dots
\end{document}